\chapter{Introduction}
%\paragraph{Scheduling on a high level}
Computer systems use resources to function. 
Often times these reources are finite and scarce. 
Scheduling can be viewed as a governance model that \textit{governs} how the resources are shared among users and how various parts of the system function together. 
The governance model is designed so as to meet goals about resource management and performance of the system. 
As an example, an operating system uses resources such as memory, CPUs, and disk.
It allows multiple processes to use these resources at the same time. 

%\paragraph{Scheduling's history}
Scheduling is a well studied problem both from theoretical and practical perspectives. 
Typically a scheduling problems is tasked with an objective and with a goal to find a schedule that can either meet the objective or get close to it.
\todo{For example, job scheduling with deadline}.
Many scheduling problems are difficult to solve (NP hard).
Prior work uses a variety of techniques to solve scheduling problems such as heuristics, dynamic programming or greedy approaches. 

%\paragraph{Scheduling for databases}
In this dissertation, we study the problem of scheduling for modern database systems.
We distinguish the traditional database systems and modern database systems based on two aspects: resource and data volumes.

Modern database systems are built for modern hardware, which includes large main memories, Non-Uniform Memory Access (NUMA) patterns and large number of CPU cores.
Traditional database architecture is built for the disk based era, when memories were smaller, data were mostly resident on disks and multi-core parallelism was uncommon. 
The distinction between traditional database architecture and modern database architecture is important from the resource landscape perspective. 
As scheduling is closely coupled to resource management, we pay attention to modern database design and 




