% !TEX root = scheduler.tex
\section{Background}\label{sec:background}
In this section we establish some prerequisites for the proposed Quickstep scheduler framework.
\sys{}~\cite{quickstep-system} is an open-source relational database engine 
designed to efficiently leverage contemporary hardware aspects such as large main memory, multi-core, and multi-socket server settings. 

The control flow associated with query execution in \sys{} involves first parsing the query, and then optimizing the query using a cost-based query optimizer.
The optimized query plan is represented as a Directed Acyclic Graph (DAG) of relational operator primitives. 
The query plan DAG is then sent to a \textit{scheduler}, which is the focus of this chapter. 
The scheduler runs as a separate thread and coordinates the execution of all queries. 
Apart from the scheduler, \sys{} has a pool of worker threads that carry out computations on the data. 

\sys{} uses a query execution paradigm, which is built using previously proposed approaches~\cite{quickstep-storage,morsel}.
In this paradigm a query is executed as a sequence of \textit{work orders}. 
A work order operates on a data block, which is treated as a self-contained mini-database~\cite{quickstep-storage}.  (c.f. Section~\ref{sec:workorders} for examples of work orders)
The computation that is required for each operator in a query plan is decomposed into a sequence of work orders. 
For example, a select operator produces as many work orders as there are blocks in the input table. 
We first describe the storage management in \sys{} in Section~\ref{sec:storage-manager}.
\sys{}'s work order abstraction (described in Section~\ref{sec:workorders}) is tied with its storage management (described in Section~\ref{sec:storage-manager}).

\section{Storage Management in \sys{}}\label{sec:storage-manager}
Data organization in the \sys{} storage manager holds the key to intra-query 
parallelism~\cite{quickstep-storage}. 
Data in a relation are organized in the form of blocks. 
Each block holds a collection of tuples from a single table. 
A unique aspect of the storage organization in \sys{} is that blocks are considered to 
be independent and self-contained mini-databases. 
Thus, when creating an index, instead of creating a global index with 
``pointers'' to the tuples in blocks, the index fragments are stored within the blocks. 
Each block is internally organized into \textit{sub-blocks}. 
There is one \textit{tuple storage sub-block}, which could be in a row store or a 
column store format.
%\reminder{Should we add more formats like compressed column store etc?}. 
In addition, each block has one sub-block for each index created on the table. 
CSB+-tree~\cite{csb+-tree} and BitWeaving~\cite{bitweaving} 
indices are currently supported. 
The blocks are free to self-organize themselves and thus a given table may have blocks in different formats. 
For example, new blocks in a table may be in a row store format, while older blocks may 
be in a column store format.

This storage block design, as articulated earlier in~\cite{quickstep-storage} enables the query execution to be broken down into a set of independent tasks on each block. 
This is a crucial aspect that we leverage in the design of our scheduler. 

The storage manager also contains a \textit{buffer pool manager}. 
It organizes the memory as an array of slots, and overlays blocks on top of the slots (so block sizes are constrained to be a multiple of the underlying slot size). 
Memory allocations for data blocks for \textit{both} permanent and temporary tables are always made from a centralized buffer pool. 
In addition, all allocations for run-time data structures, such as hash tables are also made 
by the buffer pool. 
The buffer pool manager employs an LRU-2 replacement policy. 
Thus, it is possible for a hash table to get evicted to disk, if it has become ``cold''; e.g. if it belongs to a suspended query.

\section{Work Orders}\label{sec:workorders}
Work done for executing a query in \sys{} is split into multiple \textit{work 
	orders}. %, as highlighted in Section~\ref{sec:background}. 
A work order contains all the information that is needed to process tuples in a given 
data block. 
A work order encapsulates the relational operator that is being applied, the relevant 
input relation(s), location of the input data block, any predicate(s) to be 
applied on the tuples in the input block, and descriptors to other run-time 
structures (such as hash tables).
Consider the following full table scan query to illustrate the work order concept:

\begin{lstlisting}[language=SQL, 
basicstyle=\ttfamily\small, 
showstringspaces=false,
keywordstyle=\color{cardinal}\bfseries, 
emph={San,Diego}, 
emphstyle=\color{bondiblue}\bfseries]
SELECT name FROM Employee WHERE city=`San Diego'
\end{lstlisting}	

The plan for this query has a simple \textit{selection} operator.  
For the selection operator, the number of work orders is same as the number of input blocks in the \verb+Employee+ table. 
Each selection work order contains the following information:
\begin{itemize}
	\item {Relation: \verb+Employee+, attribute: \verb|name|}
	\item {Predicate: \lstinline[language=SQL, 
		basicstyle=\ttfamily\small, 
		keywordstyle=\color{cardinal} \bfseries,
		emph={San,Diego}, 
		emphstyle=\color{bondiblue}\bfseries]|city=`San Diego'|}
	\item {The unique ID of an input block from the \verb+Employee+ table}
\end{itemize}

The work orders for a join operation %(illustrated below) 
are slightly more complicated. 
For example, a \textit{probe work order},  contains the unique ID of the probe 
block, %(as described in Section~\ref{ssec:storage-manager}, a hash table is stored in a block in the buffer pool), build and probe relations, 
a pointer to the hash table, the projected attributes, and the join predicate.
Each operator algorithm (e.g. a scan or the build/probe phase of a hash-join) 
in the system has a C++ class that is derived from a root abstract base class that has a virtual method called \verb|execute()|.
Executing a work order simply involves calling the \verb|execute()| method 
on the appropriate operator algorithm C++ class object. 
