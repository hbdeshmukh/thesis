% !TEX root = quickstep.tex

\section{Conclusions \& Future Work} \label{conclusions}
Compute and memory densities inside individual servers continues to grow at an astonishing pace. % and is likely poised to grow faster than the networks that connect nodes.
Thus, there is a clear need to complement the emphasis on ``scaling-out'' with an approach to ``scaling-up'' to exploit the full potential of parallelism that is packed inside individual servers. 
%In addition, in read-mostly data warehousing environments it is often common to use a distributed storage layer (e.g. HDFS or EBS) for fault-tolerance, which reduces the need to lean on distribution in the compute infrastructure to increase data availability. (We acknowledge that distributed execution does help query restart, but faster query execution also helps reduce the need for query restarts.) Many cloud infrastructure are getting built with a storage cloud component and a separate high-performance compute cloud component. These trends, arguably, point to a bigger emphasis on single-node compute performance especially when high single node performance can be used to reduce the overall total cost of processing queries. 


This paper has presented the design and implementation of \Quickstep\ that emphasizes a scaling-up approach. \Quickstep\ currently targets in-memory analytic workloads that run on servers with multiple processors, each with multiple cores. \Quickstep\ uses a novel independent block-based storage organization, a task-based method for executing queries,  a template metaprogramming mechanism to generate efficient code statically at compile-time, and optimizations for predicate push-down and join processing. 
We also present end-to-end evaluations comparing the performance of \Quickstep\ and a number of other contemporary systems. Our results show that \Quickstep\ delivers high performance, and in some cases is faster than some of the existing systems by over an order-of-magnitude. 

Aiming for higher performance is a never-ending goal, and there are a number of additional opportunities to achieve even higher performance in \Quickstep. Some of these opportunities include operator sharing, fusing operators in a pipeline, improvements in individual operator algorithms, dynamic code generation, and exploring the use of adaptive indexing/storage techniques. We plan on exploring these issues as part of future work. We also plan on building a distributed version of \Quickstep. 

%, and support a generic platform API that expands beyond SQL to include graph analytics, document stores and relational learning. Early prototypes show that these seemingly disparate analytic applications can be served using a platform that implements relational kernel operators~\cite{QuickFOIL, FanRP15, json-relational}. But, there are a number of issues related to optimization and execution that still need to be explored. 

%In this paper we presented the design, implementation, and evaluation of the current version of single-node \Quickstep. 
%, which uses a variety of novel techniques for data storage, indexing, and query processing for in-memory analytic query processing. 
%Key aspects of the \Quickstep\ architecture include the use of a novel block-based design for storage management and a corresponding  scheduler-based mechanism to break down a query into a sequence of work orders. This approach allows for a flexible way to adapt query execution mid-flight at run-time to use more hardware to speed up a query. \Quickstep\ also has in-built methods to effectively use NUMA hardware, which is becoming increasingly common in server configurations today. \Quickstep\ employs novel techniques such as BitWeaving to speed up scans, which are the workhorse of in-memory analytical database systems. \Quickstep\ also employs an aggressive schema-based denormalization method called WideTable to evaluate complex join queries using simple (BitWeaved) scans. Collectively, \Quickstep\ uses a set of new design points for in-memory analytical query processing, and in this paper we show how these aspects work in an end-to-end system. 

%There is a long list of next steps, including examining how to gracefully spill to disks/NVM storage, automatic physical schema tuning, building robust query optimization methods, and expanding to a distributed version of \Quickstep. % that works with the Hadoop YARN~\cite{yarn} framework. 
