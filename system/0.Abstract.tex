% !TEX root = quickstep.tex

\begin{abstract}
Modern servers pack enough storage and computing power that just a decade ago was spread across a modest-sized cluster. This paper presents a prototype system, called Quickstep, to exploit the large amount of parallelism that is packed inside modern servers. Quickstep builds on a vast body of previous methods for organizing data, optimizing, scheduling and executing queries, and brings them together in a single system. Quickstep also includes new query processing methods that go beyond previous approaches. To keep the project focused, the project's initial target is read-mostly in-memory data warehousing workloads in single-node settings. In this paper, we describe the design and implementation of Quickstep for this target application space. We also present experimental results comparing the performance of Quickstep to a number of other systems, demonstrating that Quickstep is often faster than many other contemporary systems, and in some cases faster by orders-of-magnitude. 
%We also use Quickstep for end-to-end comparison of different storage layouts, quantifying the impact of these layouts on modern systems.
%These experiments show that Quickstep is often faster than many other contemporary systems, and in some cases faster by an order-of-magnitude. 
Quickstep is an Apache (incubating) project.  
%and lives at:  \small{\url{https://github.com/apache/incubator-quickstep}}. 
%Quickstep's open-source nature implies that it can be freely used as an experimental platform by other researchers to implement their ideas, and to study the impact of their methods on end-to-end performance.

\end{abstract}
