\section{Summary of Experiments}\label{sec:experiment-summary}
We have described a large number of experiments in Section~\ref{sec:experiments} on queries from the TPC-H benchmark.
In this section, we summarize our findings and connect our results back to the original dimensions (c.f. Section~\ref{sec:dimensions}) we used for our analysis.
Recall that our focus is to understand the relative performance of pipelining strategy compared to a non-pipelining strategy.

Our high level conclusion is that in the in-memory setting, for systems using block-based architecture, the performance of pipelining and non-pipelining strategy is similar.
We now discuss the impact of individual dimensions.

\paragraph*{\textbf{Block size}}
We find that a higher block size bridges the gap between pipelining and non-pipelining. 
A larger block size results in a lower degree of parallelism for operators in a pipeline and thus also aides those operators that suffer from poor scalability. 
A very large block size however can cause memory fragmentation.
It may also result in a very low DOP such that CPU cores are left underutilized. 

\paragraph*{\textbf{Parallelism}}
Parallelism can affect the performance of pipelining and non-pipelining strategy in different ways. 
We saw in Section~\ref{sssec:parallelism-effect}, how a poorly scalable operator can benefit the performance of pipelining strategy, where as it can adversely impact the performance of pipelining strategy.

\paragraph*{\textbf{Storage Format}}
The gap between pipelining and non-pipelining performance is largely unaffected by the choice of storage format for the base tables.
We note that individually some queries execute faster when run on base tables stored in column store format. 
The benefit of using column store format over row store format is the highest for base tables (typically leaf operators in a query plan).
As the number of attributes in tables reduce from the leaf levels of the query plans to the root, the advantage of using a column store starts to diminish. 

\paragraph*{\textbf{Hardware Prefetching}}
Hardware prefetching improves pipelining query performance.
The effect is more prominent in row store than in column store format. 
We saw that prefetching improves scan performance in a representative query.
Prefetching had no effect on probe operations and it affected the performance of build hash operations adversely. 

As L3 cache size increase becomes a less viable option due to power constraints, hardware prefetching techniques are being looked at with greater interest~\cite{DBLP:journals/csur/Mittal16c}. 
Combined with the software-based prefetching efforts~\cite{DBLP:conf/icde/ChenAGM04, rof}, hardware prefetching could provide greater benefits in the future. 