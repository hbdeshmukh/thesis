\section{Pipelining Background and Related Work}\label{sec:pipe-background}
In this section we give a background about pipelining in database systems and discuss the related work. 

A simplest pipeline of relational operator consists of two operators: A producer operator and a consumer operator. 
The output of the producer operator can be passed (or \textit{streamed}) to the consume operator. 
An example of such a simple pipeline can be a hash based join. 
If there's a selection on the probe side, the selection operation and the subsequent probe operation can form a pipeline (assuming the hash table is already built).

Deeper pipelines may consist more than two operators, such that any two adjacent operators can form a producer and consumer pair.
Data from the original producer operator can be passed all the way to the last operator in the pipeline. 
An example of such deep pipeline could be a left-deep plan for a multi-way join query.

%A query plan can be viewed as set of connected pipelines. 
%Certain operators are pipeline-breakers which means data cannot be streamed beyond the pipeline breaking operator. 
%
There are two aspects about pipelining: \textit{Materialization} (or the lack of) and \textit{eager} execution of consumer operator on the output of the producer operator. 
Note that various systems may have different representations for the temporary data, which is the output of a producer operator. 
Systems such as MonetDB~\cite{monetdb} and \sys{}~\cite{quickstep-system} that operate with the block-style processing model fully materialize the output. 
Vectorwise~\cite{vectorwise} has a compact representation of the intermediate output and does not fully materialize the ouptut.
Systems such as Hyper~\cite{hyper}, LegoBase~\cite{legobase} generate compiled code for the full pipeline, therefore they do not need to have a representation for the temporary data.

Pipelining in database systems has been studied extensively. 
Most recently Wang et al.~\cite{wang2016elastic} proposed an iterator model for pipelining in in-memory database clusters.
Their key idea is to provide flexibility in the traditional iterator through operations such as expand and shrink.
Neumann~\cite{DBLP:journals/pvldb/Neumann11} proposed compilation techniques for query plans, which is used by Hyper~\cite{hyper, morsel}. 
As we mentioned earlier, query compilation is one of the techniques for realizing pipelining in a query plan. 
Vectorwise~\cite{vectorwise} pioneered the vectorized query processing model through the hyper-pipelining query execution~\cite{hyper-pipelining}.
Breaking up from the tradtitional tuple-at-a-time processing model, Vectorwise used batches (or vectors) of tuples. 
These batches could be amenable to parallel SIMD instructions and thus could provide greater speedups over Vectorwise's predecessor MonetDB~\cite{monetdb}.

Menon et al. propose Relaxed Operator Fusion model~\cite{rof} that puts together techniques like compilation, vectorization and software prefetching in a single query processing engine Peloton~\cite{pelotondb}. 

There is large body of prior work on the effect of storage format and page layouts on query performance~\cite{quickstep-storage, DBLP:journals/vldb/AilamakiDH02, DBLP:conf/vldb/HankinsP03, DBLP:journals/pvldb/GrundKPZCM10, DBLP:conf/sigmod/AbadiMH08}.
In our work, we only focus on using row store and column store format for the comparison between pipelining and non-pipelining. 

Incorporating parallelism for query execution within single node database deployment has been an active area of study since the prevalence of multi-core computing which is exemplified by many modern systems ~\cite{quickstep-system, morsel, vectorwise, hyrise-website}. 

Liu and Rundensteiner~\cite{DBLP:conf/vldb/LiuR05} study pipelined parallelism in bushy plans and propose alternatives to maximal pipeline processing. 
Our work differs from them in multiple aspects: We focus on single node in-memory query execution with large intra-operator parallelism. 
We focus on the query scheduler phase, which comes after the optimal query plan has been generated by the optimizer.
Their work focuses on optimizing query plans in the distributed execution environment with limited memory per node.

Zhu et al. propose look ahead techniques to increase robustness of query plans~\cite{DBLP:journals/pvldb/ZhuPSP17} in the Quickstep system~\cite{quickstep-system}.
Their key technique is to minimize the data that passes from the producer operator to the consumer operator in a pipeline through the help of bloom filters.

Work sharing across queries has been studied extensively. ~\cite{DBLP:conf/sigmod/HarizopoulosSA05, DBLP:conf/vldb/ZukowskiHNB07}.
Scan sharing has shown to have significant improvements in query performance, especially in the disk-setting.
Such sharing typically happens for the selection operation on large tables (e.g. lineitem in TPC-H schema), which in many query plans is the starting point of a pipeline. 

%Given the wide variety of the related work in this domain, we now concretely describe the scope of our work.
%In the in-memory single node setting on systems that use the block-based query processing, we examine the gap between pipelining and non-pipelining execution strategies. 
%The two strategies behave similarly w.r.t how they materialize the intermediate output.
%They differ based on how eagerly the intermediate data is consumed by the producer operator. 
%In the pipelining strategy, the intermediate data is consumed \textit{eagerly}, i.e. as soon as it is produced.
%In the non-pipelining strategy, the intermediate data is consumed \textit{lazily}, i.e. after all the tasks of the producer are complete.
%
%We focus on two aspects: First, whether the input to an operator is hot and second, prefetching. 
%Next, we discuss the hotness of input factor.
%
%\subsection{Hotness of input}
%The input to the probe work order is the output of the selection $\sigma$(R) and the hash table. 
%In the pipelining strategy, the input $\sigma$(R) is likely to be hot in the last level cache when the probe work order is scheduled. 
%Therefore we expect the last level cache efficiency of the probe work order to be better in the pipelining strategy than that in the no-pipelining strategy.
%
%\subsection{Data Prefetching}
%Processors use a technique called cache prefetching through which data are brought to the caches (prefetched) anticipating its need in the future. 
%Prefetching can be performed by the hardware or software.
%Software prefetching is achieved by explicitly inserting prefetch instructions in the code. 
%
%There are two kinds of hardware prefetching: temporal and spatial~\cite{hardware-prefetching}. 
%Temporal prefetching is associated with cases when the same data are accessed repeatedly.
%Spatial prefetching is performed when a sequential access pattern is likely. 
%When a block $i$ is accessed, its neighbor block $i + 1$ is prefetched.
%
%We are interested in spatial data cache prefetching.
%For selection operation that has a sequential access pattern such a prefetching could be useful .
%
%In the pipelining implementation, during selection work order execution, the L3 cache is likely to prefetch another R block (denoted by $\spadesuit$ in Table~\ref{table:pipe-vs-nopipe}).
%Such a prefetching work may be wasteful, if the next work order that gets scheduled is a probe work order. 
%Similar prefetching also happens in the no-pipelining strategy (denoted by \# in Table~\ref{table:pipe-vs-nopipe})), which is not likely to be wasted.
%This is because in the no-pipelining strategy, probe work orders do not overlap with selection work orders.