Analytical database systems process data to find insights. 
Database systems today are facing challenges from multiple fronts: First, there is an unprecedented data being generated by machines and humans.
Second, there is a diverse demand for analytics which includes relational analytics used for reporting and predictive analytics for forecasting.
Third, the hardware landscape keeps evolving which means the database system need to keep up in order to reap the best out of the hardware.
Finally cloud computing challenges the many assumptions that the traditional systems were built on with regards to deployment environments and availability of computing, storage and network resources.
Given these challenges, the modern database systems need to manage the resources at their disposal \textit{efficiently} - to produce faster results, \textit{effectively} - to fully utilize the resources and \textit{transparently} - to be accountable to its users.

In this dissertation, we present the query scheduler of Quickstep database system that imparts efficiency, effectiveness and transparency to the resource management of Quickstep. 

In the first chapter we present the makeup of the Quickstep system. 
We describe a task abstraction called Work Orders, which are managed by the scheduler.
Through work orders, Quickstep scheduler obtains fine grained control over query execution, and allows the system to exploit large amounts of parallelism offered by modern hardware.

Next we showcase the usefulness of the Quickstep scheduler as it solves two challenges: resource governance and performance.
In the second chapter, we employ the Quickstep scheduler for allocating resources such as CPU to concurrent users of the system.
The resource allocation can understand high level policies such as fair and priority-based, and enforces them while allocating the resources.
The scheduler has a learning component, which constantly monitors the resource consumption happening in the system, anticipates the future resource requirements and adjusts the resource allocation so as to meet the policy goals.
Our experiments demonstrate that we are able to meet the goals as per the policy specifications.

In the final chapter, we highlight the impact of scheduling techniques on individual query performance.
We turn to pipelining, which is a well known query processing technique, used since the times of disk-based systems.
We revisit the importance of pipelining in the in-memory systems and examine the role of various parameters such as block size, parallelism, storage format and hardware prefetching.
For queries with complex query plans, we also provide a pipeline sequencing algorithm so as to efficiently process the pipelines and complete the query execution. 